%\chapter*{Forewords}

{
\newpage
\thispagestyle{empty}
\mbox{}
\newpage
}

\pagestyle{chap}




\section*{Abstract}

\subsection*{Mining and ranking closed itemsets from large-scale transactional datasets}

\begin{paragraph}{Keywords}
  Data mining, Big Data, Parallel systems,
  Association rules,
  Quality measures
\end{paragraph}

The recent increase of data volumes raises new challenges for itemset mining algorithms.
In this thesis, we focus on transactional datasets (collections of items sets, for example supermarket tickets)
containing at least a million transactions over hundreds of thousands items.
These datasets usually follow a ``long tail'' distribution:
a few items are very frequent,
and most items appear rarely.
Such distributions are often truncated by existing itemset mining algorithms,
whose results concern only a very small portion of the available items (the most frequents, usually).
Thus, existing methods fail
to concisely provide relevant insights on large datasets.
We therefore introduce a new semantics which is more intuitive for the analyst:
browsing associations per item, for any item, and less than a hundred associations at once.
%The system should be able to answer such requests instantly, for any item,
%which suggests a batch-oriented computation of results.

To address the items' coverage challenge,
our first contribution is the item-centric mining problem.
It consists in computing, for each item in the dataset,
the $k$ most frequent closed itemsets containing this item.
We present an algorithm to solve it, \toppi.
We show that \toppi computes efficiently interesting results over our datasets,
outperforming simpler solutions
or emulations based on existing algorithms,
both in terms of run-time and result completeness.
We also show and empirically validate how \toppi can be parallelized, on multi-core machines and on Hadoop clusters,
in order to speed-up computation on large scale datasets.

%The selection of a few results of interest is also
%To address the  challenge, our second contribution
Our second contribution is \capa, a framework allowing us to study which existing measures of association rules' quality
are relevant to rank results.
This concerns results obtained from \toppi or from \jlcm,
our implementation of a state-of-the-art frequent closed itemsets mining algorithm (LCM).
Our quantitative study shows that the \nbm quality measures we compare can be grouped
into 5 families, based on the similarity of the rankings they produce.
We also involve marketing experts in a qualitative study,
in order to discover which of the 5 families we propose highlights
the most interesting associations for their domain.

Our close collaboration with Intermarch\'e,
one of our industrial partners in the \datalyse project,
allows us to show extensive experiments on real, nation-wide supermarket data.
We present a complete analytics workflow addressing this use case.
We also experiment on Web data.
Our contributions can be relevant in various other fields,
thanks to the genericity of transactional datasets.

Altogether our contributions allow analysts to discover associations of interest in modern datasets.
We pave the way for a more reactive discovery of items' associations in large-scale datasets,
whether on highly dynamic data or for interactive exploration systems.







\vfill\pagebreak
\section*{R\'esum\'e}

\begin{paragraph}{Mots-cl\'es}
Fouille de donn\'ees,
Grandes masses de donn\'ees,
Syst\`emes parall\`eles,
R\`egles d'association,
Mesures de qualit\'e
\end{paragraph}

Les algorithmes actuels pour la fouille d'ensembles fr\'equents sont d\'epass\'es par
l'aug\-mentation des volumes de donn\'ees.
Dans cette th\`ese nous nous int\'eressons plus particuli\`erement aux donn\'ees transactionnelles
(des collections d'ensembles d'objets, par exemple des tickets de caisse)
qui contiennent au moins un million de transactions portant sur au moins des centaines de milliers d'objets.
Les jeux de donn\'ees de cette taille suivent g\'en\'eralement une distribution dite en ``longue traine'':
alors que quelques objets sont tr\`es fr\'equents, la plupart sont rares.
Ces distributions sont le plus souvent tronqu\'ees par les algorithmes de fouille d'ensembles fr\'equents,
dont les r\'esultats ne portent que sur une infime partie des objets disponibles (les plus fr\'equents).
Les m\'ethodes existantes ne permettent donc pas de d\'ecouvrir des associations concises et pertinentes
au sein d'un grand jeu de donn\'ees.
Nous proposons donc une nouvelle s\'emantique, plus intuitive pour l'analyste:
parcourir les associations {\em par objet}, au plus une centaine \`a la fois, et ce {\em pour chaque objet} pr\'esent
dans les donn\'ees.

Afin de parvenir \`a couvrir tous les objets,
notre premi\`ere contribution consiste à d\'efinir la {\em fouille centr\'ee sur les objets}.
Cela consiste \`a calculer, pour chaque objet trouv\'e dans les donn\'ees,
les $k$ ensembles d'objets les plus fr\'equents qui le contiennent.
Nous pr\'esentons un algorithme effectuant ce calcul, \toppi.
Nous montrons que \toppi calcule efficacement des r\'esultats int\'eressants sur nos jeux de donn\'ees.
Il est plus performant que des solutions naives ou des \'emulations reposant sur des algorithmes existants,
aussi bien en termes de rapidit\'e que de compl\'etude des r\'esultats.
Nous d\'ecrivons et exp\'erimentons deux versions parall\`eles de \toppi
(l'une sur des machines multi-coeurs,
l'autre sur des grappes Hadoop)
qui permettent d'acc\'elerer le calcul à grande \'echelle.

Notre seconde contribution est \capa,
un syst\`eme permettant d'\'etudier quelle mesure de qualit\'e des r\`egles d'association
serait la plus appropri\'ee pour trier nos r\'esultats.
Cela s'applique aussi bien aux r\'esultats issus de \toppi que de \jlcm,
notre impl\'ementation d'un algorithme r\'ecent de fouille d'ensembles fr\'equents ferm\'es (LCM).
Notre \'etude quantitative montre que les 39 mesures que nous comparons
peuvent \^etre regroup\'ees en 5 familles,
d'apr\`es la similarit\'e des classements de r\`egles qu'elles produisent.
Nous invitons aussi des experts en marketing \`a participer \`a une \'etude qualitative,
afin de d\'eterminer laquelle des 5 familles que nous proposons met en avant
les associations d'objets les plus pertinentes dans leur domaine.

Notre collaboration avec Intermarch\'e,
partenaire industriel dans le cadre du projet \datalyse,
nous permet de pr\'esenter des exp\'eriences compl\`etes et portant
sur des donn\'ees r\'eelles issues de supermarch\'es dans toute la France.
Nous d\'ecrivons un flux d'analyse complet, \`a m\^eme de r\'epondre \`a cette application.
Nous pr\'esentons également des exp\'eriences portant sur des donn\'ees issues d'Internet;
gr\^ace \`a la g\'en\'ericit\'e du mod\`ele des ensembles d'objets,
nos contributions peuvent s'appliquer dans d'autres domaines.

Nos contributions permettent donc aux analystes
de d\'ecouvrir des associations d'objets au milieu de grandes masses de donn\'ees.
Nos travaux ouvrent aussi la voie vers la fouille d'associations interactive \`a large \'echelle,
afin d'analyser des donn\'ees hautement dynamiques
ou de r\'eduire la portion du fichier \`a analyser \`a celle qui int\'eresse le plus l'analyste.
%ou de s'int\'egrer dans des syst\`emes interactifs.




\section*{Remerciements}


Je tiens tout d'abord à remercier mes parents pour leur indéfectible soutien ces 29 dernières années.
Merci à Sihem Amer-Yahia et Vincent Leroy pour leur investissement tout au long de ma thèse,
ainsi qu'à Alexandre Termier et Marie-Christine Rousset pour m'avoir mis le pied à l'étrier scientifique.
Merci enfin à Etienne Millon de m'avoir relu et aidé à dresser \LaTeX.

\noindent
Beaucoup d'autres ont contribué de manière aussi indispensable qu'indirecte aux travaux qui suivent ;
la \textit{mixtape} p.\pageref{chap:discography} vous est dédiée.


\vfill

\begin{paragraph}{}\noindent
  Th\`ese effectu\'ee au LIG dans le cadre du \textit{programme d'investissement d'avenir} Datalyse.
  Manuscrit transmis aux rapporteurs le Jeudi 2 Juin 2016.
\end{paragraph}
